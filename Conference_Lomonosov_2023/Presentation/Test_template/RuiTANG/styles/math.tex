%!TEX root = ../RuiTANG.tex

%Поддержка математики
\usepackage{amsthm}
\usepackage{amstext}
\usepackage{amsmath}
\usepackage{amssymb}
\usepackage{amsfonts}
%Красивые дроби
\usepackage{xfrac}
%Дополнительные математические окружения
\usepackage{mathtools}
%Математический шрифт
\usepackage[bigdelims,vvarbb]{newtxmath}
%Прямые греческие буквы (принято в русской типографии)
\usepackage{upgreek}
% %Тонкое оформление перекрёстных ссылок
% \usepackage[russian]{cleveref}

%Окружения для набора утверждений и определений
\theoremstyle{plain}
\newtheorem{theorem}{Теорема}[section]
\newtheorem{lemma}{Лемма}[section]
\newtheorem{corollary}{Следствие}[section]
\newtheorem{proposition}{Утверждение}[section]
\newtheorem{theorem*}{Теорема}
\newtheorem{lemma*}{Лемма}
\newtheorem{corollary*}{Следствие}
\newtheorem{proposition*}{Утверждение}
\theoremstyle{definition}
\newtheorem{definition}{Определение}[section]
\newtheorem{remark}{Замечание}[section]
\newtheorem{example}{Пример}[section]
\newtheorem{definition*}{Определение}
\newtheorem{remark*}{Замечание}
\newtheorem{example*}{Пример}

% Нумерация формул по разделам
\numberwithin{equation}{section}
